\section{Preliminaries and motivations}
\label{sec:preliminaries}

TODO

Il s'agit ici de mettre en place les différentes briques...

\subsection{Security Design Patterns}

\sg{Cf papier \cite{silva:hal-02958111} (OLD/4-preliminaries.tex)}

\subsection{Design by Contract}

\sg{Cf papier \cite{silva:hal-02958111} (OLD/4-preliminaries.tex)}

Usage ?
Limitations ?

\subsection{Runtime enforcement mechanisms}

\sg{Des trucs intéressants à voir/reprendre dans \cite{bonfanti2023component} ?}

\subsection{Motivations} 

Ce qu'on essaie de faire ici...

Je pose quelques idées en vrac :

\begin{itemize}
    \item Travailler avec du legacy code
    \item Pas seulement safety mais security pour gérer le truc qu'on ne sait pas prévoir
    \item Dissocier les préoccupations fonctionnelles vs securité : aspect oriented programming (tissage @runtime)
    \item Faire du monitoring@runtime avec un niveau de monitoring configurable
    \item Idée d'un "harnais" de sécurité au dessus d'un code fonctionnel
    \item Idée de sécurisation de la composition de composants (eux-même peut-être déjà sécurisés): un contrat pour sécuriser le composition ?
\end{itemize}





