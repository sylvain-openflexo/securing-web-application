\begin{abstract}

TODO

%With the ever growing digitization of activities, software systems are getting more and more complex. They must comply with new usages, varied needs, and are  permanently exposed to new security vulnerabilities. 

%%Security concerns are mainly focus on cyber-attacks and so cyber-vulnerabilities but one of the main driver of such vulnerabilities is the code quality resulting of the software development process. 

%Security concerns must be addressed throughout the entire development process and in particular through appropriate architectural choices. The security patterns are the founding principles to provide the architectural and design guidelines. 
%%Actually, the security patterns encapsulates a security problem and the related solutions and are extensively defined and used in several contexts. 
%Nevertheless, researchers have pointed out the need for further research investigations to improve quality and effectiveness of security patterns. 

%In this paper, we focus on enhancing security patterns specification to improve the security of the systems using them.
%%In this paper, we target the improvement of the way in which security patterns are specified and used, and therefore, the improvement of the systems in which these patterns are used.
%we target the improvement of the legacy systems and the ensuring of the dynamic behavior of the patterns.   

%Thus, to reach this goal, we present a formal Design by Contract approach to improve the behavioral definition of the security patterns. This approach seeks to define both functional behavior and implicit parts of security design patterns. 
%%To face this behavioral definition, the contract definition is extended to take into account all the security pattern scope.
%Our approach includes the contract formalization of security patterns and a comparative implementation on two Java annotation frameworks. 
%The application of the proposal in a proof of concept case highlights the security enforcement at design time or on a legacy source code. 
%%To further prove the merits of our approach, \textit{e.g.}, its effectiveness, in-depth studies in real cases are needed. %We are currently working on such a study with a real Web application.
%The comparison of these two experiments argues that the security pattern scope is supported from design time to runtime, thus to ensure temporal properties.
 

\end{abstract}


%% Sylvain: + 2 other important items
%% - Design Patterns as a way to reuse existing software and/or components, with a focus on security concerns as exposed by the pattern. A non-intrusive approach
%% - A temporal scope to allow expression of temporal expressions (eg LTL)

