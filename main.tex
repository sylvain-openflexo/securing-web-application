
\documentclass[sigconf]{acmart}
%\documentclass[sigconf,anonymous=true]{acmart}

\usepackage{listings}
\usepackage[colorinlisttodo]{todonotes}

\usepackage{xcolor}

\definecolor{codegreen}{rgb}{0,0.6,0}
\definecolor{codegray}{rgb}{0.5,0.5,0.5}
\definecolor{codepurple}{rgb}{0.58,0,0.82}
\definecolor{backcolour}{rgb}{0.95,0.95,0.92}

\DeclareTextFontCommand{\mytexttt}{\ttfamily\hyphenchar\font=45\relax}

\lstdefinestyle{mystyle}{
    backgroundcolor=\color{backcolour},   
    commentstyle=\color{codegreen},
    keywordstyle=\color{magenta},
    numberstyle=\tiny\color{codegray},
    stringstyle=\color{codepurple},
    basicstyle=\ttfamily\footnotesize,
    breakatwhitespace=false,         
    breaklines=true,                 
    captionpos=b,                    
    keepspaces=true,                 
    numbers=left,                    
    numbersep=5pt,                  
    showspaces=false,                
    showstringspaces=false,
    showtabs=false,                  
    tabsize=2
}

\lstdefinestyle{mystyle2}{
    backgroundcolor=\color{backcolour!40},
    commentstyle=\color{codegreen!60}\rmfamily,
    keywordstyle=\color{magenta},
    numberstyle=\tiny\color{codegray},
    stringstyle=\color{codepurple},
    basicstyle=\ttfamily\scriptsize,
    breakatwhitespace=false,
    breaklines=true,
    flexiblecolumns=true,
    captionpos=b,
    emptylines=0,
    tabsize=2,
 %   keywordsprefix={@},
    texcl=true
}

%\lstset{style=mystyle2,morecomment=[s][\color{gray}]{@}{\ }}

%\lstset{language=Java,
%basicstyle=\ttfamily,
%keywordstyle=\color{javapurple}\bfseries,
%stringstyle=\color{javared},
%commentstyle=\color{javagreen},
%morecomment=[s][\color{javadocblue}]{/**}{*/},
%numbers=left,
%numberstyle=\tiny\color{black},
%stepnumber=2,
%numbersep=10pt,
%tabsize=4,
%showspaces=false,
%showstringspaces=false}

\lstset
{
    style=mystyle2,
    language=Java,
%    basicstyle=\footnotesize,
    numbers=left,
    numbersep=2pt,
    stepnumber=1,
    showstringspaces=false,
    morecomment=[s][\color{blue}]{@}{\ }
}

\pagestyle{plain}
%\pagenumbering{arabic}

%\copyrightyear{2020}
%\acmYear{2020}
\setcopyright{acmlicensed}
%\setcopyright{none}
%\acmConference[Workshop SSE '2020]{Workshop SSE '2020: International Workshop on Secure Software Engineering}{August 25--28, 2020}{Dublin, Ireland}
%\acmBooktitle{Workshop SSE '2020: International Workshop on Secure Software Engineering, August 25--28, 2020, Dublin, Ireland}
%\acmDOI{10.1145/1122445.1122456}
%\acmISBN{978-1-4503-9999-9/20/08}
%\acmPrice{15.00}

\copyrightyear{2023}
\acmYear{2023}
\setcopyright{acmlicensed}
\acmConference[CONF]{A conference}{A date}{Event place}
\acmBooktitle{An ACM book title}
\acmPrice{price}
\acmDOI{doi}
\acmISBN{isbn}

\begin{document}

\settopmatter{printfolios=true}
\settopmatter{printacmref=true}

%%
%% The "title" command has an optional parameter,
%% allowing the author to define a "short title" to be used in page headers.
\title{Securing legacy code using Security Contracts}

% Proposition de Joël: Security Contracts applied to legacy code


%%
%% The "author" command and its associated commands are used to define
%% the authors and their affiliations.
%% Of note is the shared affiliation of the first two authors, and the
%% "authornote" and "authornotemark" commands
%% used to denote shared contribution to the research.

\author{Sylvain Guérin}
\affiliation{Lab-STICC,
  \institution{ENSTA Bretagne}
  \city{Brest}
  \country{France}
}
\email{sylvain.guerin@ensta-bretagne.fr}

\author{Joel Champeau}
\affiliation{Lab-STICC,
  \institution{ENSTA Bretagne}
  \city{Brest}
  \country{France}
}
\email{joel.champeau@ensta-bretagne.fr}

\author{Salvador Martinez}
\affiliation{Lab-STICC,
  \institution{IMT Atlantique}
  \city{Brest}
  \country{France}
}
\email{salvador.martinez@imt-atlantique.fr}

\author{Henri Stoven}
\affiliation{Lab-STICC,
  \institution{ENSTA Bretagne}
  \city{Brest}
  \country{France}
}
\email{henri.stoven@ensta-bretagne.org}

%%
%% By default, the full list of authors will be used in the page
%% headers. Often, this list is too long, and will overlap
%% other information printed in the page headers. This command allows
%% the author to define a more concise list
%% of authors' names for this purpose.
\renewcommand{\shortauthors}{S. Guérin, et al.}

\newcommand*{\sm}[1]{{\color{blue}#1}}
\newcommand*{\jc}[1]{{\color{orange}#1}}
\newcommand*{\sg}[1]{{\color{magenta}#1}}

%%
%% The abstract is a short summary of the work to be presented in the
%% article.

\begin{abstract}

Security patterns represent reusable solutions and best practices aimed to avoid security related flaws in software and system designs. Unfortunately, the implementation and enforcement of these patterns remains a complex and error prone task. As a consequence, and besides implementing a given security pattern, applications often remain insecure w.r.t. the security risk they intended to tackle. This is so for two main reasons: 1) patterns are rarely re-usable without adaptation, and thus, concrete implementations may fail to deal with a number of (often implicit) properties, which must hold in order for the pattern to be effective; 2) patterns are deployed in environments with uncertainties that can only be known at runtime; 

In order to deal with this problem, we propose here \emph{Security Contracts}, a framework that permits the specification and run-time monitoring of security patterns and related properties (including temporal ones) in both, new and existing applications. It is based on an extension of the design-by-contract paradigm to enable the specification of security patterns and on the run-time adaptation of applications. We demonstrate the feasibility of our approach with a prototype implementation of our framework and its evaluation on a real use-case scenario.

\end{abstract}


%%
%% The code below is generated by the tool at http://dl.acm.org/ccs.cfm.
%% Please copy and paste the code instead of the example below.
%%

%%
%% Keywords. The author(s) should pick words that accurately describe
%% the work being presented. Separate the keywords with commas.
\keywords{Design by contract, Security Patterns, Security Contracts, Runtime monitoring}


%%
%% This command processes the author and affiliation and title
%% information and builds the first part of the formatted document.
\maketitle

\section{Introduction}
\label{sec:introduction}

TODO

\sg{Regarder ce qu'on peut reprendre comme idées du OLD/3-introduction.tex}

The paper is organized as follows. Section \ref{sec:preliminaries} presents some preliminaries on Security Design Patterns, principles of Design by Contract as well as runtime enforcement mechanisms. Then it introduces the motivations of this paper. 


\section{Preliminaries \& Running Example}
\label{sec:preliminaries}

This sections is structured as follows. First, we introduce the main concepts behind our approach, this is, the Design by Contract paradigm and security design pattern. Then, we describe in detail the \emph{Authenticator} pattern which will serve as a running example throughout the rest of the paper.

\subsection{Design by Contract}

The Design by Contract (DbC) concept was coined by Meyer \cite{meyer1992applying} as an approach to design reliable software based on the idea that elements of a software system collaborate with each other on the basis of mutual obligations and benefits, an so a contract between these elements.

Meyer indeed realized that the software systems, and in particular object-oriented systems, depend on the division of work. This means that tasks are classically divided in several sub-tasks, each being conducted by a software unit.

When this happens, the actual interaction of the units is entirely defined in a \emph{contract}. This contract contains the liabilities and benefits of the interaction for all parties involved. This analogy led Meyer to the idea of software contracting \cite{meyer1992applying}: to define contracts between clients (i.e., object’s callers) and suppliers (i.e., object functions or methods).

For software unit limited to a routine or method, a contract is defined as the aggregation of two assertions to this method:
\begin{itemize}
    \item A precondition:  Boolean condition that needs to be verified before executing the method. It summarizes the caller's obligations towards the supplier.
    \item A postcondition: Boolean condition that needs to be verified after the execution is made. It summarizes the supplier's obligations towards the caller.
\end{itemize}

In addition to these two assertions by method, Meyer defined a third type of Boolean expression called a \textit{class invariant}. This notion relies on the class concept, in object-oriented designs. A class is the representation of specific concept, usually defining its own objects or instances. Most of the time, a few properties characterize the essence of the class and should be true at all time and for every instance. A classic example of this idea, in a authentication process, a proof of identity must preserve its integrity, can't be updated. 

A proof of identity instance should verify at any time that its internal data can't be updated. This property is then an \textit{Invariant} for this class.
The whole idea of the DbC approach is that since a contract is formally defined for each unit (class) or module, bugs are unlikely to appear at run-time because of a misunderstanding between software units.



The DbC approach initially focused on software entities with limited scope, typically a class, was also extended to the notion of components \cite{beugnard1999,beugnard2010contract}.
We can also underline that this approach has been mainly used with the objective of software reliability in terms of software safety. 
The debug during the software development is the main interest of DbC. 

The security of software systems becoming a major concern at the moment, it seems relevant to evaluate this DbC approach, both in the design phase but also to guarantee a security contract during the execution.
Our objective is so to associate the Design by Contract with the Execution under Contract.

\subsection{Security Design Patterns}

In order to face the threats in terms of software security, security design patterns have naturally imposed themselves in the continuity of design patterns.
An abundant literature has been published, including books containing catalogs of patterns \cite{fernandezBooks}.

On the basis of these catalogs, different works have been developed to provide pattern languages in order to classify and create relations between patterns according to threats \cite{fernandez2013} or a formalization of patterns \cite{behrens2018} or specialized studies by domain like automotive for example \cite{cheng2019}.

Some works such as \cite{fernandez2018abstract} try to provide abstractions to the initial patterns in order to favor the expression of the semantics of the pattern. This abstract definition offers also the possibility to produce different more specialized patterns.  
We can note however that these abstract definitions respect the classical patterns approach and the expression of the solution remains on a structural definition based on a class diagram, certainly conceptual, and sequence diagrams for the behavioral part.

Like in \cite{fernandez2018abstract}, a security pattern is composed of several parts where mainly the problem, the solution and some additional comments are defined. These parts are deeply illustrated in the next section with the Authenticator pattern which is used throughout this paper to exemplify our approach and implementation.

\begin{comment}
\begin{itemize}
    \item The \emph{Intent}. This part describes in a few sentences the abstract intent of the pattern.
    \item The \emph{context}. The purpose is define the context to use the security pattern. 
    \item The \emph{problem}. This part describes the problem is addressed by the pattern.
    \item The \emph{forces}. This part lists several interested properties relative to the use of the pattern. Again this part is just a textual description

    \item The \emph{solution}. This part is usually composed of a UML (Unified Modeling Language) class diagram and a few sentences to explain the diagram.

    \item The \emph{remarks}. The authors of the pattern can add in this part any information they think is relevant. It usually involves performance results.
\end{itemize}


The following subsection presents the Authenticator pattern which is used throughout this paper to exemplify our approach and implementations.
\end{comment}

\subsection{The Authenticator pattern}

As defined in the previous section, the Authenticator pattern is characterized as follows:  
\begin{itemize}
    \item The \emph{Intent}. How to verify that a subject (user or system) intending to access the system is who it says it is? The subject must present information that is recognized by the system and is given some proof of the successful authentication.
    
    \item The \emph{context}. Computer systems contain sensible resources. We only want subjects that have some reason to be in our system to get into the system. 
    
    \item The \emph{problem}. The purpose is to prevent a malicious subject could try to impersonate a legitimate user to have access to the sensible resources. This could be particularly serious in case of the  impersonated user has a high level of privilege or offer the possibility to increase this level.

    \item The \emph{forces}. We emphasize some forces on this pattern:
    \begin{itemize}
        \item Authentication information protection. The pattern must assume the non-usurpation of the authentication information.
        % properties P1 and P2 and implicitly P3
        
        \item Authenticate authority integrity is preserve during its life to enforce the non-modification of the authentication information.
        % property P3 and implicitly P6
        
        \item Proof of identity tamper resistance. The integrity of the proof of identity presented by the user is preserved during its life-cycle.
        %Property P5 implicit P4
        
        \item Authentication process performance. Authentication should have a short response time or users will be annoyed and tried to bypass the authentication process.
        % Non property

        \item Authentication frequency. A user can't authenticate frequently in a given period of time. The purpose is to avoid many authentication tentative for a user or more annoying from security point of view, automatic attempts on the authenticate authority. 
        %P7     
    \end{itemize}

    This strengths list points to properties that must be respected to ensure proper use of the pattern.

    \item The solution. The solution definition includes a structural part and a behavioral one. The structural is generally express through a UML (Unified Modeling Language) class diagram like \ref{fig:authclasses}.
    
    Several entities are defined in this diagram:
    \begin{itemize}
        \item a \textit{Subject} needing to be authenticated,
        
        \item a \textit{Proof of Identity}, token given to the subject once the authentication is complete,
        
        \item an \textit{Authenticator} is the object which implements an authentication algorithm and creates the \textit{Proof of Identity},
        
        \item \textit{Authentication Information} are the information provided by the \textit{Subject} to the \textit{Authenticator}.
    \end{itemize}

The behavior definition is mainly based on several scenarios between the structural entities. These scenarios are illustrated through the code of our implementation, described in the section \ref{sec:case-study}.

    \item The \emph{remarks}. This last part is not detailed in this paper but usually the pattern authors add in this part any information they think is relevant. It usually involves performance results or customization aspects. 
\end{itemize}




\begin{figure}
    \centering
    \includegraphics[width=0.7\columnwidth]{figures/AuthenticatorPattern.pdf}
    \caption{UML class diagram of the Authenticator pattern}
    \label{fig:authclasses}
\end{figure}

Based on this pattern, we will demonstrated how to create a security contract with related properties and how the implementation is independent and extensible of this defined contract. Trough the definition of the Authentication contract we emphasize the creation of a security harness on an existing code in weaving in it security aspects.
With our approach, we create also a link between the development process with the Design by Contract (DbC) and the Execution under Contract (EuC), to verify at run time the security properties.


%The goal of this pattern is to verify the identity of a subject. Such a subject can be someone interacting with a software by providing credentials, a subsystem requesting data to another subsystem, a thread requesting extra privilege to an operating system, etc. The pattern only provides an architecture which is compatible with the use of every authentication algorithm. The architecture of the Authentication pattern is presented in Figure \ref{fig:auth_classes}.


% Insister en conclusion sur comment créer un contrat de sécurité à partir de ce pattern et vérifier @run time que les prop de sécurité sont vérifiées qq soitr l'implantation.
% indépendance de l'implemenattion par rapport au contrat


%\subsection{Motivations} 

%\sm{We remove this subsection. Motivation is given at the end of 2.3. We say that the pattern is abstract and needs to be adapted and given more details in order for the pattern to be effective.}



%\begin{itemize}
%    \item Travailler avec du legacy code    
%   \item Pas seulement safety mais security pour gérer le truc qu'on ne sait pas prévoir
%    \item Dissocier les préoccupations fonctionnelles vs securité : aspect oriented programming (tissage @runtime)
%    \item Faire du monitoring@runtime avec un niveau de monitoring configurable
%    \item to associate the Design by Contract with the Execution under Contract.
%    \item Idée d'un "harnais" de sécurité au dessus d'un code fonctionnel
%    \item Idée de sécurisation de la composition de composants (eux-même peut-être déjà sécurisés): un contrat pour sécuriser le composition ?
%    \item besoin de fournir des implantations efficaces et démontrables des patterns cf article bib patterns, automotive ??
%\end{itemize}







\section{Security Contracts approach}
\label{sec:security-contracts-approach}

We devote this section to the description of our \emph{Security Contracts} approach, which builds on our preliminary work in~\cite{guerin:hal-03217126}. As mentioned above, \emph{Security Contracts} is an extension of the DbC method specially tailored to the representation and monitoring of cyber security patterns. Concretely, we extend DbC so that the contracting party is composed not only of individual classes and methods, but sets of collaborating classes as we may find in security design patterns. Additionally, our extension includes tool support for the instantiation of security patterns as cross-cutting \emph{aspects} on host applications, tool support for the run-time monitoring (and enforcement) of pattern properties (i.e., EuC) and a library of re-usable security patterns. In particular, this section (i) formally describes cyber-security contracts, (ii) presents the application of such contracts to security patterns, and (iii) presents tool support and develops the case of the authenticator pattern to illustrate our approach.

\subsection{Security Contract}

%\begin{equation}
%SecurityContract = SecurityPattern + DbC
%\end{equation}


\begin{figure}
    \centering
%    \includegraphics[width=1.0\columnwidth]    
    \includegraphics[width=1.0\columnwidth]{figures/MetamodelSecurityContractV3.pdf}
    \caption{Security Contract Metamodel}
    \label{fig:securitycontractmetamodel}
 \end{figure}

 The first component of our approach is a mechanism for the specification of security patterns as contracts. We base this mechanism on a \emph{Security Contracts} metamodel which we depict in Figure~\ref{fig:securitycontractmetamodel} (note that this model is implemented as Java annotations as we explain in Section~\ref{subsec:toolsupport}). The core abstraction of our metamodel is the \emph{Contract} concept which associates formal properties with contracting parties and a run-time instance. 

%Concretely, \emph{Contracts} are associated to a number of \emph{Contracting Parties} (which can be classes, methods, instances, modules, etc), contain a number of \emph{Clauses} and are  Instances of contracts are reified (see the $\emph{ContractInstance} concept)  \emph{Clauses} represent classical DbC pre/post conditions and invariants.a
 
Concretely,  the \textit{Contract} concept has an attribute \textit{intent}, is associated with a collection of contracting parties and is composed of \textit{Clauses}. Additionally, \textit{Contracts} are \emph{realized} in a \emph{ContractInstance} so that we can represent patterns as stateful objets and follow their evolution at run-time. The \textit{intent} is a text that describes, in natural language, the purpose of the contract. The contracting parties are represented by the \textit{ContractingParty} concept and they correspond to the programming entities whose behaviors are subject to the \textit{Contract}. These entities can be classes, methods, instances, modules, etc. A \textit{Contract}  (i.e., a cyber contract) in our approach differs from the classic view of contracts proposed by Meyer in his DbC paradigm: DbC contracting parties can only be classes and methods whereas ours are not limited by their nature (instances, set of classes, programming units, etc.).. Furthermore, the *-* cardinality in the relation between the \textit{Contract} and \textit{ContractingParty} concepts enables the definition of a \textit{Contract} involving multiple \textit{ContractingParties}, potentially of different nature. This relation also provides the possibility to define multiple contracts for a given \textit{ContractingParty}. In our cyber contract context, the \textit{ContractingParties} stand for the entities involved in the security pattern definition. 

Every \textit{Contract} contains a set of \emph{Clauses}. There are three types of \textit{Clauses} that can be added to a \textit{Contract}: \textit{AssertionClause}, \textit{InvariantClause} and \textit{DelegateClause}. A property (in our case, a cyber-security pattern property) which represent properties ensured by the contract refers to one of the two first types of clauses. \textit{AssertionClauses} enable the specification of assertions that need to be verified at a certain time. They extend the notions of precondition and postcondition of the DbC approach by including a temporal aspect. Conversely, \textit{InvariantClauses} are assertions that must always be true, and can therefore be expressed with classical logic. The \textit{DelegateClause} is a concept allowing for responsibilities of a contract to be divided into one or several subcontracts. This mechanism is typically used in the object-oriented paradigm to decompose a contract on the pattern classes, or any decomposition unit of the targeted language.

Assertion and invariant \emph{Clauses} contain a \textit{BooleanExpression}. This concept is an abstraction representing any kind of logic property that one could want to enforce using a contract. In a goal to remain independent of the property expression language, the metamodel does not require any precise expression language. Note however that a constraint exists for the logic used in \textit{BooleanExpressions} within \textit{AssertionClauses}. Such logic must support time-based reasoning.

 
%On essaie de décrire l'approche SecurityContract indépendamment de tout langage de programmation (on pourrait le faire en n'importe quoi). Présenter donc les concepts et les aspects méthodologiques liés à ce qu'est un SecurityContract.
%On y trouve pêle-mêle: des aspects structurels (avec des rôles), des aspects comportementaux, et des propriétés à garantir.
%On continue en disant qu'on a choisi de faire nos expérimentations en Java, et qu'on a choisi Pamela.


\subsection{Contract specification example}
\label{subsection:secureapplication}

%The implementation of the \textit{Authenticator} pattern on the secured application is relatively transparent to the application developer, from a functional point of view. However, the run-time weaving mechanism now explicitly manages %instances of the \textit{Authenticator} pattern (\textit{AuthenticatorPatternInstance}), with their own life cycle that evolves as sessions are created and deleted in the web application. The \textit{SessionInfo} class is now equipped with an %authentication mode, insofar as the methods annotated with \textit{@RequiresAuthentication} (e.g. method \texttt{checkSecure()}, line 50 of the listing \ref{listing:SessionInfo}) can only be executed if the current instance is declared as the %subject of an instance of the \textit{Authenticator} pattern for which authentication has been satisfied. This allows to protect any unauthorized and malicious code call in the context of a vulnerability that would give access to the %execution of this code.a

Let’s exemplify the definition of a security contract for the Authenticator pattern. To do so, we will first use formal boolean expressions in order to describe the contract properties. Note that these boolean expressions suppose the existence of a object oriented representation of the pattern as depicted in Figure~\ref{fig:authclasses} (e.g., we suppose the existence of Classes representing the concepts of the authenticator pattern such as \emph{Subject}). The \textit{Authenticator} pattern intrinsically defines six security properties (four implicit properties and two functional properties), which have been previously introduced as pattern forces in Section~\ref{sec:preliminaries}: 

\begin{enumerate}
    \item \textbf{Unicity of the couple subject/authentication information}
    \begin{equation*}
        P1: \forall a, b \in I_{Subject}, a \ne b \implies a.authInfo \ne b.authInfo
    \end{equation*}
    
    \item \textbf{Authentication information integrity }
    \begin{equation*}
        P2: \forall a \in I_{Subject}, a.authInfo = a.authInfo_{ini}
    \end{equation*}
    
    \item \textbf{Authenticate authority integrity}
    \begin{equation*}
        P3: \forall a \in I_{Subject}, a.authenticator = a.authenticator_{ini}
    \end{equation*}

    \item \textbf{Verification of the validity or an undefined proof of identity}
    \begin{equation*}
      \begin{split}
        &P4: \forall a \in I_{Subject}, (a.idProof = \emptyset) \lor \\&(a.idProof = a.authenticator.request(a.authInfo))
      \end{split}
    \end{equation*}

    \item \textbf{Continue verification of the proof of identity}
    \begin{equation*}
        P5: self.idProof = self.authenticator.request(self.authInfo)
    \end{equation*}
\textbf{Remark:}After authentication, the proof of identity must always match the value initially returned by the query method. The joint verification of properties P4 and P5 guarantees the validity of the proof of identity during the entire session.

    \item \textbf{Post-condition for the method\textit{request(authInfo)}}
    \begin{equation*}
        P6: self.check(authInfo) \lor returnValue = \emptyset
    \end{equation*}
 \end{enumerate}


With these properties and the abstract definition of the Authenticator pattern in Figure~\ref{fig:authclasses} we can define our \emph{authenticator} security contract (note that this pattern is extended with an additional property in Section~\ref{sec:case-study}). Figure \ref{fig:cybercontractXML} shows a simplified XML representation of this contract. This representation illustrates an instantiation of our metamodel on the Authenticator cyber contract. The binding with the Authenticator pattern is explicit and we can see the property P1 as an \textit{InvariantClause} defined at the pattern scope. The delegation mechanism provided with the \textit{DelegateClauses} is also illustrated within this example. Some of the previous properties are indeed within the scope of a single method (P5 and P6). They are thus delegated to the relevant classes using a \textit{DelegateClause} (keyword \textit{Subcontract}) and then to the correct methods.

\begin{figure}
    \centering
 \begin{lstlisting}[breaklines=true, language=XML, basicstyle=\ttfamily\footnotesize, mathescape=true]
<Contract>
    <binding "Authenticator_pattern" />  
    <clauses> 
    <InvariantClause P1/>  
    <Subcontract "SubjectContract">
        <binding "Subject" />
        <InvariantClause P2 $\land$ P3 $\land$ P4/>
        <Subcontract "authenticateContract"s>
            <binding "void authenticate()">
            <ensures P5/>
        </Subcontract>
    </Subcontract>
    <Subcontract "AuthenticatorContract">
        <binding "Authenticator"/>
        <Subcontract "requestContract">
            <binding = "ProofOfIdentity request(AuthenticationInformation authInfo)">
            <ensures P6/>
        </Subcontract>
    </Subcontract>
    </clauses>
</Contract>

\end{lstlisting}
\caption{XML representation of the Authenticator pattern contract}
\label{fig:cybercontractXML}
\end{figure}


\subsection{Deployment, Monitoring and Enforcement}

We have described above a metamodel allowing us to describe security patterns as contracts in an abstract manner. For these contracts to have a practical use we need the means to: 1) deploy them in host applications which need to be secured; and 2) monitor then at run-time to both, verify and enforce the contract properties. 

For the operationalization of our \emph{Security Contracts} we choose to follow an Aspect Oriented Approach. Indeed, security mechanisms constitute a classical example of cross-cutting concern and thus, AoP appears as a natural choice. In our approach, abstract \emph{Security Contracts} as defined above are translated to a sort of \emph{advices} which encapsulate the properties to monitor and enforce at run-time. Details of this process are given in Section~\ref{subsec:toolsupport} where we describe a prototype implementation of our approach for the Java ecosystem.


\subsection{Tool Support}
\label{subsec:toolsupport}

In order to demonstrate the feasibility of our approach, a prototype implementation have been developed for the Java ecosystem. It builds on top of the PAMELA framework~\cite{guerin:hal-03217126}. PAMELA is a Java modelling framework which focuses on bridging the gap between software modelling and code implementation. In this context, PAMELA modelling framework proposes an approach where models are directly weaved in source code by means of Java annotations. Thus, it does not require code generation. That way, both model and source-code coexist in the same artifact.

\begin{figure}
    \centering
    \includegraphics[width=1.0 \columnwidth]{figures/PamelaVisionV2.pdf}
    \caption{Overview of the PAMELA approach}
    \label{fig:PamelaVision}
\end{figure}

Figure \ref{fig:PamelaVision} illustrates the PAMELA framework where we have at design time (left side of the figure) the model weaved in source code files based on the Java annotations and at run-time (right side of the figure) the PAMELA interpreter ensuring the link between the model behavior and the application Java byte-code. Resulting execution is a combination of (i) executing plain Java byte-code as the result of the basic compilation of source code, and (ii) an embedded PAMELA interpreter executing PAMELA model semantics. From an implementation standpoint, PAMELA uses \emph{javassist} reflection library including the \texttt{MethodHandler} mechanism. Based on this support, the Java dynamic binding is overridden to provide the call of the PAMELA model behavior when an object method is invoked. 

In our current scenario, \emph{security contracts} become PAMELA models to be weaved in host applications. We need however to extend the PAMELA framework to include our notion of \emph{Pattern}, i.e. a composition of multiple classes, known as \emph{Stakeholders}, whose expected behavior is defined in a pattern contract, along with formal properties which must be ensured at run-time. More specifically, implementing \emph{Patterns} with PAMELA provides:
\begin{itemize}
    \item the ability to monitor the execution of the code;
    \item the ability to offer extra structural and behavioral features, executed by the PAMELA interpreter;
    \item a representation of \emph{Patterns} as stateful objects. Such objects can then evolve throughout run-time and compute assertions using any paradigms (e.g., LTL - Linear Temporal Logic);
    \item the ability of having multiple extension points. In other words, the ability to create new patterns, or to redefine or specialize existing ones.
\end{itemize}

\emph{Patterns} are defined in PAMELA using three classes, each one representing a different conceptual level:
\begin{itemize}
    \item a \mytexttt{PatternFactory}. This class is responsible for identifying, at run-time, the declared patterns in the Java byte-code.
    \item a \mytexttt{PatternDefinition}. This class represents an occurrence of the pattern in the supplied byte-code. It has the responsibility of maintaining links with all classes and methods involved in the pattern, as well as managing the life-cycle of its \mytexttt{PatternInstances}.
    \item a \mytexttt{PatternInstance}. This class represents the instance of a pattern at run-time. It is responsible for maintaining the pattern state and providing pattern behavior and contract enforcement.
\end{itemize}

To declare a \emph{Pattern} on existing code, pattern elements such as \emph{Pattern Stakeholders} and methods need to be annotated with provided pattern-specific  annotations. These annotations will be discovered at run-time by the \mytexttt{PatternFactory} and stored in \mytexttt{PatternDefinition} attributes.



\subsubsection{The \textit{Authenticator} Security Contract in PAMELA}
\label{subsec:AuthenticatorSecurityContract}

In order to illustrate the main features of our prototype, we present here how our running example, i.e., the \emph{Authenticator} pattern is implemented with PAMELA and its extension for security contracts. In that sense, Figure~\ref{fig:PAMELAAuthenticatorCD} presents the PAMELA class structure for security contracts instantiated for the  Authenticator pattern. Note that each attribute of the \mytexttt{AuthenticatorPatternDefinition} class has a corresponding annotation (displayed as a comment). Then, the code excerpt in Figure~\ref{fig:ExampleOfAuthenticatorPattern} shows how the Authenticator pattern is used by means of annotations to an existing base of code in which an instance of the \texttt{Manager} class plays the \emph{Authenticator} role, while an instance of the \texttt{Client} class plays the \emph{Subject} role. At run-time, both functional code and security pattern logic, i.e pattern behavior and contract enforcement, are weaved by the PAMELA framework. 
More specifically, \mytexttt{PatternDefinition} classes implement the \mytexttt{isMethodInvolvedInPattern} which returns \mytexttt{true} for all methods which need to be handled by the associated \mytexttt{PatternInstances}. This allows, for instance, for the pattern contract properties to be checked at run-time before and/or after any method of interest. In the case of the Authenticator pattern, the \texttt{AuthenticatorPatternInstance} objects will check the invariants of the pattern contract before and after every call to \emph{Subject} and \emph{Authenticator} classes. These checks are performed by the \texttt{checkInvariant} method. These mappings are summarised in Figure~\ref{fig:AuthenticatorPattern}. 

\begin{figure}
    \centering
    \includegraphics[width=1.0 \columnwidth]{figures/AuthenticatorPattern4.pdf}
    \caption{PAMELA vision of the authenticator pattern}
    \label{fig:AuthenticatorPattern}
\end{figure}

Note that the user only needs to annotate his code and PAMELA will automatically handle the pattern business logic (both, pattern behavior and assertion checking). For more details about this example we direct the reader to our previous publication~\cite{silva2020contract}.

\begin{comment}
\begin{figure}
    \centering
    \includegraphics[width=1.0\columnwidth]{figures/AuthenticationClassDiagram.pdf}
    \caption{UML class diagram of the Authenticator security contract}
    \label{fig:authenticatorClassDiagram}
 \end{figure}
\end{comment}

\begin{figure}
    \centering
    \includegraphics[width=0.8 \columnwidth]{figures/PAMELAAuthenticator_CD.pdf}
    \caption{PAMELA Authenticator pattern class diagram.}
    \label{fig:PAMELAAuthenticatorCD}
\end{figure}

\begin{comment}
    
\begin{figure}
    \centering
    \includegraphics[width=1.0 \columnwidth]{figures/PAMELAAuthenticator_CD.pdf}
    \caption{PAMELA vision of the authenticator pattern}
    \label{fig:AuthenticatorPattern}
\end{figure}
\end{comment}




\begin{figure}
    \centering
\begin{lstlisting}[language=Java,basicstyle=\ttfamily\footnotesize]
@ModelEntity 
@Authenticator(patternID = "PatternId")
public class Manager {

	@RequestAuthentication(patternID = "PatternId")
	public int request(@AuthenticationInformation(
        patternID = "PatternId", paramID = "id") String id) {
		return ...;
	}
}

@ModelEntity
@AuthenticatorSubject(patternID = "PatternId")
public class Client {

	public Client(Manager authenticator, String id) {
		...
	}

	@AuthenticationInformation(patternID = "PatternId", paramID = "id")
	public String getAuthInfo() {
		return ...;
	}

	@ProofOfIdentityGetter(patternID = "PatternId")
	public int getIDProof() {
		return ...;
	}

	@AuthenticatorGetter(patternID = "PatternId")
	public Manager getManager() {
		return ...;
	}

	@AuthenticateMethod(patternID = "PatternId")
	public void authenticate() {
		setIDProof(getManager().request(getAuthInfo()));
	}

	@RequiresAuthentication
	public void securedMethod() {
		...
	}
}
\end{lstlisting}
    \caption{Example showing how to use the \textit{Authenticator} defined as a PAMELA security contract}
    \label{fig:ExampleOfAuthenticatorPattern}
\end{figure}


%\begin{figure}
%    \centering
%    \includegraphics[width=1.0 \columnwidth]{figures/AuthenticatorControlFlow.pdf}
%    \caption{Authenticator control flow}
%    \label{fig:AuthenticatorControlFlow}
%\end{figure}


\subsubsection{Security Contracts Library}

One of the strengths our approach is the separation between the definition of the abstract pattern and their deployment on a host application, as this permits the re-use of well tested and verified patterns. In this sense, our tool support includes a library of off-the-shelf patterns ready to be deployed by application developers in their source code. As of now this library includes the following patterns:

\begin{itemize}
    \item Authenticator
    \item Authorization
    \item Single access point
    \item Owner
    \item Role-based access control
\end{itemize}

Details about this patterns, their features and how to use them are available in the project's web site\footnote{https://www.openflexo.org/pamela/docs/category/security-features/}. More patterns will be added in the near future as we continue developing our framework.



\section{Case study : applying security contract to existing code}
\label{sec:case-study}

Following the definition and prototyping of \textit{security contracts} \cite{silva2020contract}, we experimented our approach on a significant use case in the field of web programming. This domain is particularly exposed to attacks but is also the target of particular attention in terms of software development, precisely to resist attacks.

To demonstrate the effectiveness of \textit{security contracts}, we thought that applying them to an existing and widely used framework would and widely used, would allow to validate our approach.

We selected the Open Source \textit{Spring} framework due to the used worldly for web's application backend. This intensive use is supported by Java language with its strong typed system.
Currently securing legacy code remains a challenge for many reasons like understanding the code architecture, identifying the security threats relative to the code and reduce the attack surface, with a first step owning a robust authentication process.

The Open Source status of this framework provides an excellent use case to tackle our Security Contracts related to both, the reuse of legacy code and on the Authentication process to ensure an efficient first shield. 


\subsection{The \textit{Spring} framework}

\textit{Spring} Spring is a framework for developing Web applications in Java used intensively.  This framework is based on predefined libraries including reusable classes. The variety of classes is both an advantage to offer the developer many possibilities of extension but also a disadvantage which implies that Spring is long to assimilate at first sight. In terms of security, these extension possibilities can be problem sources in terms of understanding and by untimely addition of bugs or vulnerabilities. 
Spring has a web module that supports the Servlet API (which dynamically creates data within an HTTP server) called Spring MVC for Model View Controller.

\begin{comment}
    
The figure \ref{fig:ArchitectureSpring} presents the architecture of the framework, the numbers of the next paragraph defining a generic scenario on this architecture.
The \textit{DispatcherServlet} of \textit{Spring} manages all the requests received by the application (1), it will first solicit the handler mapping (2) which will make the link with a registered controller bean annotated by the controller. This one will then execute the application logic provided by the \textit{HandlerAdapter} from which a model will result. It will also return a logical view name to the \textit{HandlerAdapter} (3). After that, the \textit{DispatcherServlet} will call the \textit{ViewResolver} (4) which will resolve the view thanks to the logical name (5). The  \textit{DispatcherServlet} then sends the rendering process to the returned view which will constitute, with the model, the answer to the initially received request (6) .

\jc{Pb des figures illisibles, a refaire dans drawio}
\begin{figure}[!h]
    \centering
    \includegraphics[width=1.0 \columnwidth]{figures/ArchitectureSpring.png}
    \caption{Architecture of the \textit{Spring} framework}
    \label{fig:ArchitectureSpring}
\end{figure}
\end{comment}


\textit{Spring Security} is the library allowing the management of authentication and access control. Like any \textit{Spring} project, it is customizable. 
Authentication being the first security rampart of the application, the issue here is to make sure that this \textit{Authenticator} design pattern is correctly implemented and preserves the desired security properties.
This main motivation has guided our experiment to focus on the application of Security Contracts on the authentication process.

The figure \ref{fig:ArchitectureAuthenticationSpring} details the authentication management within the \textit{Spring} framework. We explain a generic scenario in the figure via the diagram numbers. So when the application developed with \textit{Spring Security} receives a request, a component chain is activated (1) . When the request contains an authentication one, the \textit{AuthenticationFilter} will extract the user's credentials (usually username and password) and create an \textit{Authentication} object. If the received information contains a username and a password, a \textit{UsernamePasswordAuthenticationToken} will be created containing the username and the password (2) . 

This token will be used to invoke the \textit{authenticate()} method of the \textit{AuthenticationManager} which is implemented by \textit{ProviderManager} (3). There are several \textit{AuthenticationProvider} already configured and listed in \textit{ProviderManager}. The one we will use in this experimentation is the \textit{DAOAuthenticationProvider}. DAO stands for "Data Access Object", which is a model that provides an abstract interface to a database type. By mapping application calls to the persistence layer, the DAO provides certain specific data operations without exposing the details of the database. The \textit{DAOAuthenticationProvider} uses \textit{UserDetailsService} (5) to retrieve user data based on the user's username (6) , (7) , (8) , (9). If the authentication (10) succeeds then the complete \textit{Authentication} object (with "authenticated = True", the list of authorities and the user name) is returned. Finally, the \textit{AuthenticationManager} returns the \textit{Authentication} object to the \textit{Authenticationfilter}, the authentication has succeeded and the object is stored in the \textit{SecurityContext}.
And if the authentication fails,  the \textit{AuthenticationManager} raises an exception \textit{AuthenticationException} will be thrown. 

\begin{figure}[!h]
    \centering
    \includegraphics[width=1.0 \columnwidth]{figures/AuthenticateSpringProcess.pdf}
    \caption{Authenticate service in the \textit{Spring} framework}
    \label{fig:ArchitectureAuthenticationSpring}
\end{figure}

\subsection{Applying the \textit{Authenticator} Security Contract on the \textit{Spring} authentication}

% PAMELA: un framework de tissage au run-time permettant de faire du monitoring de propriétés de sécurité

The implementation of the Security Contract \textit{Authenticator} in this context is based on two phases: 1) Weaving the \textit{Authenticator} pattern into the \textit{Spring} architecture, presented just below 2) Allocating the security properties, presented in the section \ref{subsection:secureapplication}, to be preserved by the application.


These two steps are ensured by source code annotation techniques. These techniques are used by the application developer which must know the contract definition, and the application code reused.
First of all, this developer requires the identification of the appropriate concepts and extension points. Sometimes for a reuse perspective inheritance and delegation are used to expand these extension points.

Figure \ref{fig:authclasses} shows the class diagram of the \textit{Authenticator} contract presented in section \ref{subsec:AuthenticatorSecurityContract}. 
%We also refer to figure \ref{fig:AuthenticatorPattern} for the correspondence of the concepts between the business concepts provided by \textit{Spring} and the \textit{Authenticator} contract used.
This contract identifies four distinct entities: \textit{Authenticator}, \textit{Subject}, \textit{AuthenticationInformation} and \textit{ProofOfIdentity}, for which we need to find the corresponding business concept in the source code to be annotated. To examplify our approach, we select only the \textit{Authenticator} and the \textit{Subject} entities.



\subsubsection{Authenticator}: This concept is identified in the \textit{Authenticator} pattern with the \texttt{@Authenticator} annotation. On the application to secure, the Java interface \textit{AuthenticationProvider} plays the corresponding role. So this role is applied, line 3 of the listing \ref{listing:CustomAuthenticationProvider}, to the specialization \textit{CustomAuthenticationProvider} which extends \textit{AuthenticationProvider}. 

From an implementation point of view, the role is implemented by the \textit{CustomAuthenticationProviderImpl} class, presented on line 17 in the listing \ref{listing:CustomAuthenticationProvider}.
This class is extended from \textit{DAOAuthenticationProvider} class which takes care of this role in the Spring framework. 
%To specialize this logic, we use inheritance on the Spring framework API, by defining an API/implementation pair (\textit{DaoAuthenticationProvider} / \textit{CustomAuthenticationProviderImpl}) presented on line 17 in the listing \ref{listing:CustomAuthenticationProvider}. 

The link between the role and the pattern definition is done via the identifier \texttt{SessionInfo.PATTERN_ID}, which is currently a character string, see line 3.


\begin{lstlisting}[language=Java,basicstyle=\ttfamily\footnotesize, caption=Specialization of \textit{Authenticator} entity through \texttt{CustomAuthenticationProvider} definition,label=listing:CustomAuthenticationProvider]
@ModelEntity
@ImplementationClass(CustomAuthenticationProviderImpl.class)
@Authenticator(patternID = SessionInfo.PATTERN_ID)
@Imports(@Import(SessionInfo.class))
public interface CustomAuthenticationProvider extends AuthenticationProvider {
  String USER_NAME = "userName";

  // Inherited from AuthenticationProvider API
  public Authentication authenticate(Authentication authentication) throws AuthenticationException;
    
    ...
    
  abstract class CustomAuthenticationProviderImpl extends DaoAuthenticationProvider implements CustomAuthenticationProvider {

    @Override
    public Authentication authenticate(Authentication authentication) throws AuthenticationException {
      ...
    }

    @Override
    public UsernamePasswordAuthenticationToken request(String userName) {
      ...
    }
  }
}
\end{lstlisting}

\subsubsection{Subject}: 
This concept is identified in the \textit{Authenticator} pattern with the \texttt{@AuthenticatorSubject} annotation, line 3 of the listing \ref{listing:SessionInfo}. On the application to secure, this concept is not directly reified but corresponds to the notion of session, from our point of view. Thus we reify the concept \texttt{SessionInfo}, which is presented in the listing \ref{listing:SessionInfo}. 
The \textit{SessionInfo} concept allows the management of the \texttt{USER_NAME}, \texttt{IP\_ADDRESS} and \texttt{ID\_PROOF} (proof of identity) from line 5 to 10, as well as the access to the authenticator via the accessors \texttt{getAuthenticationProvider()} and \texttt{setAuthenticationProvider()}

The method \texttt{authenticate()} (line 18 of the listing \ref{listing:SessionInfo}) is identified via the annotation \texttt{@AuthenticateMethod} to the method \texttt{request} of the pattern \textit{Authenticator} as presented in the figure \ref{fig:AuthenticatorPattern}.
The \texttt{checkSecure()} method is annotated as \texttt{@RequiresAuthentication}, line 20-21,  to trigger the Authentication process based on the method with the annotation \texttt{@AuthenticateMethod} if the process is executed for the first time. 

\begin{lstlisting}[language=Java,basicstyle=\ttfamily\footnotesize, caption= \texttt{SessionInfo} to implement the \textit{Subject} entity,label=listing:SessionInfo]
@ModelEntity
@ImplementationClass(SessionInfo.SessionInfoImpl.class)
@AuthenticatorSubject(patternID = SessionInfo.PATTERN_ID)
public interface SessionInfo {

	String SESSION_INFO = "SESSION_INFO";
	String PATTERN_ID = "AuthenticatorPattern";
	String USER_NAME = "username";
	String IP_ADRESS = "ipAdress";
	String AUTHENTICATION_PROVIDER = "authenticationProvider";
	String ID_PROOF = "idProof";

    ....
    
	@Getter(AUTHENTICATION_PROVIDER)
	@AuthenticatorGetter(patternID = PATTERN_ID)
	CustomAuthenticationProvider getAuthenticationProvider();

	@Setter(AUTHENTICATION_PROVIDER)
	void setAuthenticationProvider(CustomAuthenticationProvider val);

    ....
    
	@AuthenticateMethod(patternID = PATTERN_ID)
	void authenticate();

	@RequiresAuthentication
	void checkSecure();

	abstract class SessionInfoImpl implements SessionInfo {

        ...
        
		@Override
		public void checkSecure() {
			System.out.println("checkSecure() for " + this);
		}

		@Override
		public String getIpAdress() {
			return ((ServletRequestAttributes)RequestContextHolder.currentRequestAttributes())
			           .getRequest().getRemoteAddr();
		}
	}

}
\end{lstlisting}

\begin{comment}

\subsubsection{AuthenticationInformation}: The information at the base of the authentication in the \textit{Spring} application to be secured is the user name defined in the \textit{SessionInfo} class (listing \ref{listing:SessionInfo}). A character string implements the user name and its accessed by \texttt{getUserName()} and \texttt{setUserName(String)} defined respectively lines 15 and 18 of the listing \ref{listing:SessionInfo}. The annotation \texttt{@AuthenticationInformation} defined on line 14 specifies the use of this data for authentication, together with the annotation \texttt{@AuthenticationInformation(paramId=...)} defined on line 14 of the listing \ref{listing:CustomAuthenticationProvider}.

\subsubsection{ProofOfIdentity}: The fourth and last entity to be identified is the notion of proof of identity which is found in the \texttt{UsernamePasswordAuthenticationToken} class in the \textit{Spring} application. We have chosen to expose the accessors getIDProof() and setIDProof(UsernamePasswordAuthenticationToken) in the SessionInfo interface (respectively lines 27 and 31 of the listing \ref{listing:SessionInfo}.


\subsubsection{Life cycle of the \texttt{AuthenticationProvider} concept}:
We have used the notion of \textit{Service} offered by the \textit{Spring} framework to implement a service dedicated to authentication. The listing \ref{listing:AuthManagerService} presents this implementation. This service is responsible to instantiate a single instance (singleton design pattern) of \texttt{CustomAuthenticationProvider} for the application. As well, this service creates a session with an instance of (\texttt{SessionInfo}), via the use of a factory based on the construction of the \texttt{PamelaMetaModel} inferred from the \texttt{CustomAuthenticationProvider} class (lines 11 and 12 of the listing \ref{listing:AuthManagerService}). 


\begin{lstlisting}[language=Java,basicstyle=\ttfamily\footnotesize, caption=Mise en oeuvre du service d'authentification : \texttt{AuthManagerService.java},label=listing:AuthManagerService]
@Service
public class AuthManagerService {

	private CustomAuthenticationProvider authenticationProvider;

	private PamelaModelFactory factory;

	public AuthManagerService() {
		PamelaMetaModel pamelaMetaModel;
		try {
			pamelaMetaModel = new PamelaMetaModel(CustomAuthenticationProvider.class);
			factory = new PamelaModelFactory(pamelaMetaModel);
		} catch (ModelDefinitionException e) {
			e.printStackTrace();
		}

		authenticationProvider = factory.newInstance(CustomAuthenticationProvider.class);
	}

	public CustomAuthenticationProvider getAuthenticationProvider() {
		return authenticationProvider;
	}

	public SessionInfo makeNewSessionInfo() {
		SessionInfo returned = factory.newInstance(SessionInfo.class);
		returned.setAuthenticationProvider(getAuthenticationProvider());
		return returned;
	}
}
\end{lstlisting}

\subsubsection{Lifecycle of the session concept by the \texttt{SessionInfo}}: 
We used the \textit{Spring} component \texttt{HttpSessionEventPublisher} as a support for extension points allowing to manage the construction and destruction of instances of \texttt{SessionInfo}. We propose in the listing \ref{listing:CustomHttpSessionEventPublisher} a class \texttt{CustomHttpSessionEventPublisher} which inherits from \texttt{HttpSessionEventPublisher} and which implements these functionalities of management of instances of \texttt{SessionInfo} (lines 12 and 21 respectively for the creation and the destruction of instances).


\begin{lstlisting}[language=Java,basicstyle=\ttfamily\footnotesize, caption=Cycle de vie des \texttt{SessionInfo} : \texttt{CustomHttpSessionEventPublisher.java},label=listing:CustomHttpSessionEventPublisher]
@Component
public class CustomHttpSessionEventPublisher 
extends HttpSessionEventPublisher {

	@Autowired
	private AuthManagerService authManagerService;

	@Override
	public void sessionCreated(HttpSessionEvent event) {
		super.sessionCreated(event);
		if (event.getSession().getAttribute(SessionInfo.SESSION_INFO) == null) {
			SessionInfo sessionInfo = authManagerService.makeNewSessionInfo();
			event.getSession().setAttribute(SessionInfo.SESSION_INFO, sessionInfo);
		}
	}

	@Override
	public void sessionDestroyed(HttpSessionEvent event) {
		super.sessionDestroyed(event);
		// delete the session
    ((SessionInfo)event.getSession().getAttribute(SessionInfo.SESSION_INFO)).delete();
	}
}
\end{lstlisting}
\end{comment}


\subsubsection{Authentication management}: 
The management of authentication itself is intricate because of two competing levels: the authentication provided by the \textit{Spring} framework and the one implemented in our \textit{Authenticator} pattern. The alignment of the two mechanisms takes place in the implementation of \texttt{CustomAuthenticationProviderImpl} (listing \ref{listing:CustomAuthenticationProviderImpl}). The \textit{Spring} framework receives the authentication requests via the call of the method \texttt{authenticate(Authentication)} of \texttt{AuthenticationProvider}. We overwrite this method, line 13,  with the management of the current session information (from line 16 to 23), and the  \texttt{authenticate()} and  \texttt{checkSecure()} of the  \texttt{SessionInfo} entity, line 24 to 27, are called to complete the process.
%but especially by filling a table of correspondence between the username used as authentication information and the proof of identity returned by the \texttt{AuthenticationProvider} (an instance of \texttt{UsernamePasswordAuthenticationToken}). The implementation of the \texttt{request(String)} method is limited to returning this same value.

\begin{lstlisting}[language=Java,basicstyle=\ttfamily\footnotesize, caption=Authentication management with \texttt{CustomAuthenticationProvider} definition,label=listing:CustomAuthenticationProviderImpl]
abstract class CustomAuthenticationProviderImpl extends DaoAuthenticationProvider implements CustomAuthenticationProvider {

    private Map<String, UsernamePasswordAuthenticationToken> tokens 
        = new HashMap<>();
        
	/**
      * Implementation is here trivial as we use map filled by
      * {@link #authenticate(Authentication)} method
		 */
	@Override
	public UsernamePasswordAuthenticationToken request(String userName) {
		return tokens.get(userName);
	}

	@Override
	public Authentication authenticate(Authentication authentication) throws AuthenticationException {

		try {
			System.out.println("authenticate(Authentication) called for " + authentication);
			UsernamePasswordAuthenticationToken returned = (UsernamePasswordAuthenticationToken) super.authenticate(authentication);
			String name = authentication.getName();
			WebAuthenticationDetails details = (WebAuthenticationDetails) authentication.getDetails();
			String userIp = details.getRemoteAddress();
			SessionInfo sessionInfo = SessionInfo.getCurrentSessionInfo();
			sessionInfo.setUserName(name);
			sessionInfo.setIpAdress(userIp);
			tokens.put(name, returned);
			// Call authenticate() to complete process
			sessionInfo.authenticate();
			// Ensure that we are now in authenticated context
			sessionInfo.checkSecure();
			return returned;
		} catch (AuthenticationException e) {
			throw e;
		} catch (ModelExecutionException e) {
			e.printStackTrace();
			throw new SessionAuthenticationException("Exception during authentication: " + e.getMessage());
        }

	}
}
\end{lstlisting}


\subsection{Pattern \textit{Authenticator} specialisation to extend the behavior with a temporal logic property}


% discours sur ok on a définit le contrat et maintenant on étend avec une nouvelle prop et on customize la classe du framework.




One of the interests of our approach lies in the ability of the framework to provide extension points. In our case, the \texttt{CustomAuthenticationProvider} class specializes our implementation of the \textit{Authenticator} pattern.
            

 The second feature of our framework that we exploit, is the reification of the notion of instance of the \textit{Authenticator} pattern through the instance of the \texttt{AuthenticatorPatternInstance} class, see figure \ref{fig:AuthenticatorPattern}. 
This class encapsulates the instances of the subject (class \texttt{SessionInfo}) and authenticator (class \texttt{CustomAuthenticationProvider}). 
%L'exécution de ces classes est tissée au run-time. 
%The execution of these classes is woven into the run-time. 
As this class gives access to the instances of the classes of the pattern, an introspection capacity of the current state of the pattern is provided during all its life cycle. 
% et donc de l'exécution de code métier lié à l'assemblage de ces classes.

We have taken advantage of these two aspects to specialize the \textit{Authenticator} pattern by extending the behavior with a new temporized temporal property. This property takes into account a time quantity applied on an execution path of the pattern. Classically this property expresses the fact that there cannot be more than three authentication failures in a given period of time. If three failures occur in this time, the application will have to switch to another mode, classically refusing any authentication attempt for a certain time for example.

Let $auth\_fail_i$ be an "authentication failure" event in the current execution trace:

    \begin{equation*}
       \begin{split}
           P7:& \forall \{ auth\_fail_i,auth\_fail_{i+1},auth\_fail_{i+2} \} \in execution\_trace \\
          &auth\_fail_{i+2}.time - auth\_fail_i.time < TIME\_LIMIT
      \end{split}
    \end{equation*}


The class \texttt{CustomAuthenticatorPatternInstance}, listing  \ref{listing:CustomAuthenticatorPatternInstance}, implements the definition of the property \textit{P7} through the method \texttt{checkRecentAuthFailCountLessThan3} from lines 7 to 15.
This property is based on the evaluation of the \texttt{events} variable, defined line 3, which is defined as an instance variable of this class.  This variable \texttt{events} embodies the current state of the class's instances and takes part of the global state of the pattern. 
%This pattern state consists of the internal state of each class instance of the pattern. 
In our example, this variable is updated, line 17, to take into account each authentication failure and so assessed for the property evaluation.    

After the definition of the contract property, the listing \ref{listing:TemporalPropertyDefinition} presents the definition of the \textit{Authenticator} referring to this property as a precondition of the method \texttt{authenticate()} (line 9). Also we can see between the line 4 and 8, the use of the annotation \texttt{@OnException} which allows to automatically generate events corresponding to the failure of connection. This events is stored in the pattern state via the \texttt{events} variable as we described before.

%first the three listing classes \ref{listing:CustomAuthenticatorPatternFactory}, \ref{listing:CustomAuthenticatorPatternDefinition} and \ref{listing:CustomAuthenticatorPatternInstance}. 




\begin{comment}
\begin{lstlisting}[language=Java,basicstyle=\ttfamily\footnotesize, caption=Spécialisation de la classe \texttt{AuthenticatorPatternFactory.java},label=listing:CustomAuthenticatorPatternFactory]
public class CustomAuthenticatorPatternFactory extends AuthenticatorPatternFactory {

	public CustomAuthenticatorPatternFactory(PamelaMetaModel metaModel) {
		super(metaModel);
	}

	@Override
	protected Class<CustomAuthenticatorPatternDefinition> getPatternDefinitionClass() {
		return CustomAuthenticatorPatternDefinition.class;
	}

	@Override
	protected CustomAuthenticatorPatternDefinition getPatternDefinition(String patternId, boolean createWhenNonExistant) {
		return (CustomAuthenticatorPatternDefinition) super.getPatternDefinition(patternId, createWhenNonExistant);
	}
}
\end{lstlisting}




\begin{lstlisting}[language=Java,basicstyle=\ttfamily\footnotesize, caption=\texttt{CustomAuthenticatorPatternDefinition} specializes \texttt{AuthenticatorPatternDefinition},label=listing:CustomAuthenticatorPatternDefinition]
public class CustomAuthenticatorPatternDefinition extends AuthenticatorPatternDefinition {

	public CustomAuthenticatorPatternDefinition(String identifier, PamelaMetaModel pamelaMetaModel) {
		super(identifier, pamelaMetaModel);
	}

	@Override
	public <I> void notifiedNewInstance(I newInstance, ModelEntity<I> modelEntity, PamelaModel model) {
		if (modelEntity == getSubjectModelEntity()) {
			// We create a new PatternInstance for each new instance of subjectModelEntity
			CustomAuthenticatorPatternInstance<?, I, ?, ?> newPatternInstance = new CustomAuthenticatorPatternInstance(this, model,
					newInstance);
		}
	}
}
\end{lstlisting}

\end{comment}



\begin{lstlisting}[language=Java,basicstyle=\ttfamily\footnotesize, caption=\texttt{AuthenticatorPatternInstance} specialization to add the P7 property implementation,label=listing:CustomAuthenticatorPatternInstance]
public class CustomAuthenticatorPatternInstance extends AuthenticatorPatternInstance<CustomAuthenticationProvider, SessionInfo, String, UsernamePasswordAuthenticationToken> {

	public static long TIME_LIMIT = 180000; // 180s = 3 min

	private List<PatternInstanceEvent> events = new ArrayList<>();

	public CustomAuthenticatorPatternInstance(CustomAuthenticatorPatternDefinition patternDefinition, PamelaModel model, SessionInfo subject) {
		super(patternDefinition, model, subject);
	}

	// Perform check that last 3 AuthFailEvent in current execution trace were not raised within allowed time limit
	public boolean checkRecentAuthFailCountLessThan3() {
		long currentTime = System.currentTimeMillis();
		if (events.size() >= 3 && (currentTime - events.get(events.size() - 3).getDate()) < TIME_LIMIT) {
			// 3 attempts or more in TIME_LIMIT interval
			return false;
		}
		return true;
	}

	public void generateAuthFailEvent() {
			events.add(new AuthFailedEvent());
    }

	@Override
	public void authenticationSuceeded() {
		super.authenticationSuceeded();
	}
 }
\end{lstlisting}


The extension of the previous contract being realized, we can now use it and add an \texttt{@ensures} annotation with a reference to the implementation of the new property, line 9 of the listing \ref{listing:TemporalPropertyDefinition}.
This annotation is defined related to the \texttt{authenticate()} method.
So for now, any call of this method respects the security contract defined by the instance \texttt{CustomAuthenticatorPatternInstance} corresponding to the current execution. This contract automatically handles the \texttt{AuthFailedEvent} events, and checks that a user cannot fail to authenticate more than 3 times in a given time. 




\begin{lstlisting}[language=Java,basicstyle=\ttfamily\footnotesize, caption=Contract definition with applying with the P7 property,label=listing:TemporalPropertyDefinition]
public interface CustomAuthenticationProvider extends AuthenticationProvider {

    ...
    
	@Override
	@OnException(
			patternID = SessionInfo.PATTERN_ID,
			onException = AuthenticationException.class,
			perform = "patternInstance.generateAuthFailEvent()",
			strategy = OnExceptionStategy.HandleAndRethrowException)
	@Ensures(patternID = SessionInfo.PATTERN_ID, property = "patternInstance.checkRecentAuthFailCountLessThan3()")
    public Authentication authenticate(Authentication authentication) throws AuthenticationException;

    ...
}
\end{lstlisting}


Our use case shows how to extend the \textit{Authentication} process, based on an existing code, to ensure a security contract that verifies the properties P1-P7 defined at design time. Note that the full code of this case study can be found at the web site of the project~\footnote{https://github.com/openflexo-team/pamela/tree/2.0/pamela-spring-security-uc}.

With this experiment, we sought to demonstrate:
\begin{itemize}
    \item the definition of a security contract allows us to highlight the properties we seek to preserve, independently of the implementation.
    \item this definition allows to increase the level of abstraction of the classical definition of security patterns, even on a conceptual class diagram.
    \item these contracts allow the link between the Security by Design and Execution under Contract phases, which remains a problem not enough covered at present, without code generation.
    \item The use case treated also shows that the weaving of a security contract by specialization of an existing pattern, is possible and easily usable based on code annotation.
    \item the specialization of the security contract can also be progressive by extending the number of properties and specializing the code defining the contract. 
\end{itemize}



\section{Discussion and lessons learned}
\label{sec:discussion-lessons-learned}

Quelques idées en vrac aussi:

\begin{itemize}
    \item Bibliothèque de patterns > composition de patterns (pattern=contrat)
    \item Les patterns exposent des "roles" utilisables pour construire des propriétés (en tant qu'expression)
    \item Parler de paradigmes multiples pour les propriétés des patterns/contrats
    \item Sinon problème de charge cognitive pour le développeur
    \item Problème de performances
\end{itemize}

L'expérimentation menée valide l'approche proposée d'annotation de code source existant, pour spécifier une exécution composite via du tissage au runtime. Cette approche s'apparente à la programmation par aspect (AOP).

Cette approche requiert cependant une grande expertise en programmation pour les développeurs, et s'avère parfois assez technique. A ce titre, il était intéressant de partir d'un code source réaliste et représentatif de pratiques de programmation usuelles, à savoir ici le framework Spring. Il est important de mentionner qu'il n'est pas question de discuter ici de la politique de sécurité du framework Spring pour laquelle beaucoup de mécanismes adhoc, complexes et efficaces sont mis en oeuvre par la communauté. Il s'agissait uniquement ici de valider le fait qu'il est possible d'annoter du code existant pour réaliser du monitoring @runtime.

Pour valider plus complètement l'approche, il serait utile de s'intéresser à d'autres bases de code source, pour explorer plus exhaustivement des techniques de programmation permettant d'intégrer des points d'extension. 

Cette approche, de par sa nature "programmation orientée aspect", requiert une grande expertise en terme d'analyse fonctionnelle et de maintenance de code source, et implique une assez lourde charge cognitive. Ceci s'explique par la nature intrinsèquement transversale des concepts manipulés. Le développeur doit avoir en tête à la fois le code fonctionnel, et les patrons utilisés (dont la logique métier est implicite et n'est pas accessible directement au niveau du code annoté).

Le cas d'utilisation étudié ici ne porte pas d'exigences fortes en terme de performances, mais il est évident que le monitoring implique un overhead significatif du temps de calcul. Il serait donc pertinent de compléter cette expérimentation par une étude sur les performances. A noter cependant que le framework Pamela permet un ajustement fin et programmatiquement accessible des fonctionnalités de monitoring (cf section \ref{subsec:conf-monitoring}).


\section{Related work}
\label{sec:related-work}

TODO

\sg{Cf papier \cite{silva:hal-02958111} (OLD/8-relatedworks.tex)}



\section{Conclusion\& Future Work}
\label{sec:conclusion}

In this paper we have presented \emph{Security Contracts}, a novel extension of the design by contract paradigm aimed at supporting security patterns. Our approach provides a mechanism for the specification of re-usable and extensible abstract patterns, their deployment on host applications and their monitoring at run-time (in what we call, Execution under Contract). A prototype implementation for the Java ecosystem and its application to a case study involving the enhancement of the authentication mechanism provided by the Spring Security framework are presented as well. Concretely, we have shown how we can define the Authenticator pattern as a \emph{Security Contract} in an abstract way, deploy it by the means of annotations in both new and existing applications and monitor and enforce its properties (including temporal ones) at run-time.

As future work we envision the exploration of the following research lines:

\begin{itemize}
\item Pattern composition. We intend to investigate an extension of our framework in order to give support to the composition of patterns. This will enable the possibility of creating complex patterns as a composition of simpler, easier to verify ones.
\item Annotation enhancement. We aim to research the feasibility of the integration of a given (temporal) logic directly in the annotations used to deploy the pattern, so that the user can add/modify its properties.
\item Pure run-time weaving. We plan to extend our prototype so that it uses a weaving model to identify pointcuts and joins, so that the pattern's behaviour weaving can be done directly on the bytecode and at run-time without accessing the source code of the application
\end{itemize}


%%
%% The next two lines define the bibliography style to be used, and
%% the bibliography file.
\bibliographystyle{ACM-Reference-Format}
\bibliography{biblio}

\end{document}
\endinput
%%
%% End of file `sample-sigchi.tex'.
