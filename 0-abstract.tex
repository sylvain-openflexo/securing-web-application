\begin{abstract}

Security patterns represent reusable solutions and best practices aimed to avoid security related flaws in software and system designs. Unfortunately, the implementation and enforcement of these patterns remains a complex and error prone task. As a consequence, and besides implementing a given security pattern, applications often remain insecure w.r.t. the security risk they intended to tackle. This is so for two main reasons: 1) patterns are rarely re-usable without adaptation, and thus, concrete implementations may fail to deal with a number of (often implicit) properties, which must hold in order for the pattern to be effective; 2) patterns are deployed in environments with uncertainties that can only be known at runtime; 

In order to deal with this problem, we propose here \emph{Security Contracts}, a framework that permits the specification and run-time monitoring of security patterns and related properties (including temporal ones) in both, new and existing applications. It is based on an extension of the design-by-contract paradigm to enable the specification of security patterns and on the run-time adaptation of applications. We demonstrate the feasibility of our approach with a prototype implementation of our framework and its evaluation on a real use-case scenario.

\end{abstract}
