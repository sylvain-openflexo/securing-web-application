
\section{Conclusion}
In this paper, we presented a design by contract approach to formalize security patterns. This novel approach improves the behavioral definition of the security patterns and therefore enforces the security of systems at the source code level. 

The ambition of this article was to address two issues regarding the security of code through the use of security patterns. Our formalization of both functional and implicit parts of security design patterns allowed us to address the first challenge. Our contracts thus provides an explicit description of security patterns which captures tacit elements of the pattern, usually undocumented. The second targeted issue is largely addressed through our PAMELA experiment. We indeed provided an implementation allowing for run-time contract enforcement based on code annotation. Such active monitoring contributes to ensuring pattern security properties in legacy code.

%One strength of the approach presented in this paper is that it formalizes the definition of both functional behaviors and implicit parts of security design patterns. The latter is even more important since this tacit knowledge is usually not documented in security pattern definitions and therefore not always known by developers.

%The second issue that we have identified regarding the security of legacy code, is largely addressed through our PAMELA experiment based on code annotation and run-time contract enforcement.

%A second strength is the ease of implementing the proposal in annotational languages such as JML and PAMELA. 

% Sylvain: surtout dire qu'on le fait à partir d'une base de code fonctionnel existante, avec des composants logiciels qui n'ont pas été développé en prenant en compte des préoccupation de sécurité, c'est à mon avis ca le principal intérêt de l'approche.

%Even if the application of the proposal in a proof of concept case highlights the security improvement on  legacy source code, further research is needed to prove the merits of our approach; \textit{e.g.}, its effectiveness and its scalability to real cases. 

As a future work, we plan to investigate about the use, strengths and weaknesses of our approach with real cases (e.g., Web applications exposed to cyber attacks) to evaluate its scalability and effectiveness.